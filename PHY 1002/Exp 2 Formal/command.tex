
\documentclass[12pt]{article}
\usepackage{graphicx}
\usepackage{makecell}
\usepackage{float}
\newcommand{\diff}{\,\mathrm{d}}
\usepackage[margin=1in]{geometry}
\usepackage{fancyhdr}
\pagestyle{fancy}
\usepackage{extarrows}
\usepackage{breqn}
\usepackage[colorlinks,linkcolor=blue]{hyperref}
\newcommand{\N}{\mathbb{N}}
\newcommand{\Z}{\mathbb{Z}}
\newcommand{\trans}{^{\mathrm T}}
\usepackage[table]{xcolor}
\usepackage{bm}
\usepackage{array}
\usepackage[english]{babel}
\usepackage{natbib}
\usepackage{url}
\graphicspath{{images/}}
\usepackage{parskip}
\usepackage{fancyhdr}
\usepackage{vmargin}
\usepackage[font={bf, footnotesize}, textfont=md]{caption}
\usepackage{amsmath,amsthm,amssymb}
\usepackage{indentfirst}
\usepackage{bookmark}
\usepackage{fontspec}
\setmainfont{TeX Gyre Termes}
\setsansfont{TeX Gyre Heros}
\setmonofont{JetBrains Mono NL}
\usepackage{mathtools}
\usepackage{unicode-math}
\setmathrm{XITS Math}
\setmathsf{XITS Math}
\usepackage{listings}
\usepackage{ctex}

% 用来设置附录中代码的样式

\lstset{
    basicstyle          =   \ttfamily,
    keywordstyle        =   \ttfamily,
    commentstyle        =   \rmfamily\itshape,
    stringstyle         =   \ttfamily,
    % flexiblecolumns,
    numbers             =   left,
    showspaces          =   false,
    numberstyle         =   \ttfamily,
    showstringspaces    =   false,
    captionpos          =   t,
    frame               =   lrtb,
}

\lstdefinestyle{Python}{
    language        =   Python,
    basicstyle      =   \ttfamily,
    numberstyle     =   \ttfamily,
    keywordstyle    =   \color{blue},
    commentstyle    =   \color{green}\ttfamily,
    breaklines      =   true,
    % columns         =   fixed,
    basewidth       =   0.5em,
}





\newenvironment{theorem}[2][Theorem]{\begin{trivlist}
                \item[\hskip \labelsep {\bfseries #1}\hskip \labelsep {\bfseries #2.}]}{\end{trivlist}}
\newenvironment{lemma}[2][Lemma]{\begin{trivlist}
                \item[\hskip \labelsep {\bfseries #1}\hskip \labelsep {\bfseries #2.}]}{\end{trivlist}}
\newenvironment{exercise}[2][Exercise]{\begin{trivlist}
                \item[\hskip \labelsep {\bfseries #1}\hskip \labelsep {\bfseries #2.}]}{\end{trivlist}}
\newenvironment{reflection}[2][Reflection]{\begin{trivlist}
                \item[\hskip \labelsep {\bfseries #1}\hskip \labelsep {\bfseries #2.}]}{\end{trivlist}}
\newenvironment{proposition}[2][Proposition]{\begin{trivlist}
                \item[\hskip \labelsep {\bfseries #1}\hskip \labelsep {\bfseries #2.}]}{\end{trivlist}}
\newenvironment{corollary}[2][Corollary]{\begin{trivlist}
                \item[\hskip \labelsep {\bfseries #1}\hskip \labelsep {\bfseries #2.}]}{\end{trivlist}}
\DeclareMathOperator{\tr}{tr}
\DeclareMathOperator{\rank}{rank}
\DeclareMathOperator{\Span}{span}
\DeclareMathOperator{\row}{row}
\DeclareMathOperator{\col}{col}
\DeclareMathOperator{\range}{range}
\DeclarePairedDelimiterX{\inp}[2]{\langle}{\rangle}{#1, #2}
\DeclareMathOperator{\Proj}{Proj}
\DeclareMathOperator{\trace}{trace}
\newcommand{\Her}{^{\mathrm H}}
\DeclareMathOperator{\diag}{diag}
\makeatletter
\newcommand\fcaption{\def\@captype{table}\caption}
\makeatother
\setmarginsrb{3 cm}{2.5 cm}{3 cm}{2.5 cm}{1 cm}{1.5 cm}{1 cm}{1.5 cm}


\makeatletter
\let\thetitle\@title
\let\theauthor\@author
\let\thedate\@date
\makeatother

\pagestyle{fancy}
\fancyhf{}
\rhead{\theauthor}
\lhead{\thetitle}
\cfoot{\thepage}
